To conclude the whole research we can say, that overlapping data and imbalance significantly impeded the model construction.
Now let us review the dataset and draw meaningful summary of that work.

Firstly, exploratory data analysis clearly shows that, although the number of car crashes is at its height during the day,
one is less likely to end up being killed at noon. 
However, the chance of a tragic outcome increases almost fourfold at night.
Clustering car accidents also allows to obtain information on the most risky places.
Those places turn out to be junctions or motorways with a heavy traffic.

Secondly, the model for predicting the number of deaths in car accidents was constructed,
emulating the yearly seasonality and clear downward trend.
Based on comparison the best model turned out to be Prophet with Bayesian inference and MCMC simulations.

Thirdly, as we can see the existing data is insufficient to determine the outcome of a car accident.
It does not mean, that road or weather conditions have no influence on the outcome of a crash,
but that their impact is minor. At least there are other factors that are relevant and are not present in the dataset.
And that is actually true from the practical point of view.
The dataset has no information concerning the type of a vehicle, its conditions and safety equipment.
However, these things are highly important as they determine the extent to which driver is protected.
Moreover, we examined only the given features, but there is an option to construct some new features out of them.
For example one could create some combinations of features or use kernels to perform a non-linear transformation of given information.

So, based on the existing information it is impossible to meaningfully predict the outcome of a crash.
Therefore another research is needed to extract additional information and build a better model .

