Car accidents may occur due to many reasons. And usually there are multiple factors determining the outcome of a particular crash.
Among these factors are time of a day, weather conditions, car model, etc.
The outcome of a crash, either tragic one or without any loss of lives and severe injuries,
is supposed to correlate with the type of violation that caused the accident.
The rationale underneath is straightforward: as it happens, the accident occurs after someone violates the law and causes that accident.
If the violation is minor, for example, parking in the wrong place, it hardly can cause any significant injury,
however other violations, such as crossing the solid line and driving onto the opposite strip may be the cause of significant injuries and deaths.
So, the main question of this paper is to find out the extent to which factors like whether
conditions, road conditions, type of violations do correlate with the severity of car accident and
can be used as predictors to determine the outcome of a crash. \\
To answer the stated question, we examine the data on car accidents and car crashes in Moscow, collected from 2015 to 2021.
The dataset contains information on more than 55000 accidents with all the required information.
We carefully preprocess the dataset to group the features and the whole procedure of data preprocessing is described in chapter 3. \\
First and foremost, it is important to define a target variable.
And as far as the aim of the research is to determine the influence of the factors mentioned above on the outcome of the crash,
we should carefully pick the variable we will attempt to predict. There are three potential candidates: «deaths», «injuries», «severity».
In the end «Severity» was chosen as a target variable and there are two reasons for that.
Firstly, deaths are not a good target, as we cannot properly compare many injuries and one death without referring to severity.
Moreover, we cannot even compare no deaths and one injury as well.
Because one injury may be minor or hard, some need to refer to severity of the accident as well.
Secondly, following the same logic we cannot properly use injuries as an indication of the significance of the accident,
because it is impossible to directly infer the significancy of wounds.
Based on that logic «severity» was picked as a target variable, and it was preprocessed assigning «0» to light wounds, «1» to major wounds and deaths.
It is worth noting that we do not discriminate between major wounds and deaths because the data does not allow us to properly distinguish two cases.
Difference may exist because the ambulance was a couple minutes faster in one case or due to other factors omitted in the initial data.
And as two categories are quite close to each other in terms of given features we are going to treat them as one case. \\
In the end the problem is defined as classification of cases into two groups.
However, before diving into classification problem we perform the analysis of the target variable.
If you look at the variable “severity” plotted monthly, you will see a clear downward trend.
Nevertheless, it is clear that «severity» is an aggregate variable for «deaths» and «injured».
So, we look at the behaviour of the two latter indicators.
As it turns the rationale for the number of severe accidents going down is the decline in death rate.
So, me construct several models to analyze that trend and emulate the behaviour. 
The detailed description of that model is shown in the fifth chapter. \\
Afterwards, we turn to classification problem. As it will be shown later in chapter four,
we face a serious problem with imbalance in data accompanied by a significant overlap of two classes.
In order to obtain meaningful result, we perform several models. First, we use simple logistic regression and $\verb|CatBoostClassifier|$ with class weights.
Second, we implement three models specifically designed to deal with imbalanced classes: One-Class SVM, Isolation Forest and Local Outlier Factor.
The detailed description of the outcome is presented in chapter five. 
