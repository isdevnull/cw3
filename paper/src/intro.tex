"Vision Zero" is a set of innovative road safety policies, aimed at reducing traffic accidents, which result in severe injuries and deaths.
This strategy had been proposed by Swedish Road Administration in 1995 \cite{Vision0}, has been employed since then in Decade of Action for
Road Safety declared by United Nations in 2010 \cite{Vision0} and is believed to be the reason of why Scandinavian countries maintain one of the lowest
death and injury rates in car accidents in the whole world\footnote{Data is given for 2015
\href{https://www.who.int/violence_injury_prevention/road_safety_status/2015/TableA2.pdf?ua=1}{in the following table}.}.
"Vision zero" suggests relying on vast amount of data to take effective measures towards the goal,
e.g. select dangerous junctions and lanes, build safer roads, decrease speed limits, exclude human error,
separate roads from pedestrians and cyclists. Each country employs its own policy with regards to "Vision Zero". \\
In Russia "Vision Zero" is being implemented in a form of the national project\footnote{
\href{https://bkdrf.ru/}{Национальный проект "Безопасные и качественные дороги"}}, however the results are quite unsatisfactory yet:
despite number of severe injuries and deaths decrease in Moscow (which in fact does not develop any strategies according to the national project),
all other regions do not exhibit positive results. In our paper we try to explore factors that contribute to mortality rates in capital region.
